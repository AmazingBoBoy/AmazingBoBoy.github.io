\documentclass{article}


% This is the key line to make sections numbered simply as 1, 2, 3...
\renewcommand{\thesection}{\arabic{section}}

% Optional: If you want subsections to be 1.1, 1.2, etc. (relative to the section, not chapter)
\renewcommand{\thesubsection}{\thesection.\arabic{subsection}}

% --- NEW: For enumerate numbering ---
% This redefines the first level of enumeration to be SectionNumber.ItemNumber
\renewcommand{\theenumi}{\textbf{\thesection.\arabic{enumi}}}
% This ensures the label uses the new numbering format
\usepackage{enumitem} % Recommended package for list customization
\setlist[enumerate,1]{label=\theenumi} % Apply the label format to the first level of enumerate
% --- Preamble: Global Settings and Packages (as discussed previously) ---
\usepackage[utf8]{inputenc}
\usepackage[T1]{fontenc}
\usepackage{lmodern}

\usepackage{amsmath}
\usepackage{amssymb}
\usepackage{amsthm}
\usepackage{mathtools}
\usepackage{bm}
\usepackage{geometry}
\usepackage{lipsum}
\usepackage{tcolorbox}
\tcbuselibrary{listings,breakable}
\usepackage{xcolor} % Make sure xcolor is loaded for color definitions
\DeclareMathOperator{\Ima}{Im}
\usepackage{graphicx}
\graphicspath{{assets/}} % Path to global image assets
\usepackage{mathrsfs} % Add this line to your preamble for \mathscr support; tlmgr install jknapltx rsfs server-side
\usepackage{etoolbox}
\usepackage{lipsum}
\usepackage{cancel}

\setlength{\marginparwidth}{2cm} %for todonotes
\usepackage{todonotes}

\usepackage{quiver}

\usepackage[
    colorlinks=true,
    linkcolor=blue,
    urlcolor=blue,
    citecolor=green,
    pdftitle={Jiawen Huang's selection},
    pdfauthor={Your Name},
    pdfsubject={Mathematics Solutions},
    pdfkeywords={Math, Solutions, Textbook, LaTeX}
]{hyperref}

\tcbset{
  mysolutionbox/.style={
    coltitle=black,
    colback=gray!10,
    colframe=gray!30,
    boxrule=0.5pt,
    arc=4pt,
    left=5pt, right=5pt, top=5pt, bottom=3pt,
    breakable,
    title={Solution},
    fonttitle=\bfseries,
    % after upper={\par\hfill\qedsymbol\par}, % <-- Removed the square
    before upper={
      \setlength{\parskip}{0.3\baselineskip}
      \setlength{\parindent}{0pt}
      \setlist{
        nosep,
        leftmargin=1.5em,
        itemsep=0.15pt
      }
    },
    % after={
    %   \parskip=0pt
    %   \parindent=0pt
    %   \parskip=0pt
    %   \parindent=1em
    % }
  }
}

% Now, the 'solution' environment is much simpler, just invoking tcolorbox
\newenvironment{solution}{
  \begin{tcolorbox}[mysolutionbox]
}{
  \end{tcolorbox}
}


% Custom theorem/solution environments
\newtheorem{theorem}{Theorem}[section]
\newtheorem{proposition}[theorem]{Proposition}
\newtheorem{lemma}[theorem]{Lemma}
\newtheorem{definition}[theorem]{Definition}
\newtheorem{example}[theorem]{Example}
\newtheorem{remark}[theorem]{Remark}
\usepackage{hyperref}
\begin{document}
\section*{Introduction}
I am a math major at University of Califronia, Berkeley. I love math. 
\\
This document is a selction of my beloved mathematcial ideas, techniques and problems. 
\section{June 2025}
\begin{enumerate}
    \item $A$ is a $5 \times 5 $ Matrix. If $A^{10}=0$, prove $A^5=0$
    \begin{solution}
        We prove $\Ima A^5=0$.
        \\
        
        Case 1: If $\dim (\Ima A)=5$, then $\dim (\Ima A^{10})=5$, contradiction. 
        \\
        
        Case 2: If $\dim (\Ima A)\leq 4$. Note that if there exists $n$ such that $\dim (\Ima A^n)=\dim (\Ima A^{n+1})$, then $\dim (\Ima A^m)=\dim (\Ima A^{n})$ for all $m\geq n$. 
        \\
        But $\dim (\Ima A^{10})=0$, so $\dim (\Ima A^{n})$ must be strictly decreasing until it reaches to $0$. Thus, $0\le\dim (\Ima A^{5})<\dim (\Ima A^{4})<\cdots <5$. So $\dim (\Ima A^{5})$ must be $0$. 
    \end{solution}
    
    \item Nullity of a matrix (geometric multiplicity)  $\leq$ number of zero Eigenvectors (algebraic multiplicity)
    \begin{solution}
    \href{https://math.stackexchange.com/a/3241493/1139215}{https://math.stackexchange.com/a/3241493/1139215}
    \end{solution}
    \begin{remark}
        The idea is to construct a similar matrix with an easy characteristic  polynomial.
    \end{remark}
    \item $G$ is a group and n is a positive integer. Let $H$ be a subgroup of $G$ generated by all elements of order $n$ in $G$. Prove that $H$ is a normal group. 
    \begin{solution}
    Key 1: If $x^n=e$, then for any $g \in G$, $(gxg^{-1})^n=e$
    \\
    Key 2: For any $y \in H$, $y=h_1 \cdot h_2 \cdot \cdots \cdot h_k$. $gxg^{-1}=(gh_1g^{-1})(gh_2g^{-1}) \cdots (gh_kg^{-1})$. This is clearly in $H$. Thus, $gHg^{-1}=H$ for all $g \in G$. $H$ is a normal group. 
    \end{solution}
    \item Let $F=F(A)$ be the free group of $A$. Prove $F/[F,F]\cong F^{ab}(A)$
    \begin{solution}
        \[\begin{tikzcd}[column sep=tiny]
	&& {\text{commutator subgroup of F(A)}:F/[F,F]} \\
	\\
	{\text{Free group of A}: F(A)} &&&& \text{abelian}: G \\
	\\
	&& \text{any set}: A
	\arrow["{\exists! \Phi}"', from=1-3, to=3-5]
	\arrow["{\text{projection}:\pi}"', from=3-1, to=1-3]
	\arrow["{\exists !\phi}"{description, pos=0.4}, from=3-1, to=3-5]
	\arrow["{\pi \circ j}"{description, pos=0.3}, from=5-3, to=1-3]
	\arrow["j"', from=5-3, to=3-1]
	\arrow["{\text{any function}: f}", from=5-3, to=3-5]
\end{tikzcd}\]
    Because initial in category $Ab^{A}$ is unique up to isomorphism, $F/[F,F]\cong F^{ab}(A)$. 
    \end{solution}
    \begin{remark}
        At this point, I realize how powerful category theory really is! This is crazy. 
    \end{remark}
    \end{enumerate}
\end{document}
